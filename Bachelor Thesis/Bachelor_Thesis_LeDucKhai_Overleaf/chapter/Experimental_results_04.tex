\section{Encoder and initialization comparison}


\subsection{Encoder comparison}

\begin{table}[!h]
\captionsetup{font=Large}
\centering
\begin{adjustbox}{width=0.8\columnwidth,center}
\begin{tabular}{|c|c|c|c|c|} 
\hline
\multirow{2}{*}{Architecture} & \multicolumn{2}{c|}{Pre-training}                              & \multicolumn{2}{c|}{WER [\%]}  \\ 
\cline{2-5}
                              & Data                                   & Hours                 & Hykist dev & Hykist test       \\ 
\hline
\textit{Base}                 & \multirow{2}{*}{None}                  & \multirow{2}{*}{-}    & 35.8       & 39.9              \\ 
\cline{1-1}\cline{4-5}
\textit{Large}\textsubscript{1-8}             &                                        &                       & 35.0       & 40.7              \\ 
\hline
\textit{Base}                 & \multirow{2}{*}{Viet. in-house}        & \multirow{2}{*}{219}  & 30.2       & 33.3              \\ 
\cline{1-1}\cline{4-5}
\textit{Large}\textsubscript{1-8}             &                                        &                       & 31.5       & 33.4              \\ 
\hline
\textit{Base}                 & \multirow{2}{*}{Multilingual in-house} & \multirow{2}{*}{1168} & 26.2       & 28.8              \\ 
\cline{1-1}\cline{4-5}
\textit{Large}\textsubscript{1-8}             &                                        &                       & 26.8       & 28.7              \\
\hline
\end{tabular}
\end{adjustbox}
\caption{\center{Comparison for \textit{Base} and \textit{Large}\textsubscript{1-8} using different pretraining schedules: no pretraining, pretraining on in-house data and on multilingual data.
All fine-tunings are done until full convergence on Vietnamese in-house data and the recognition is done on HYKIST.}}
\label{table:encoder_compare_pretrain}
\end{table}

We compare the performance of 2 types of encoder: \textit{Base} and \textit{Large}\textsubscript{1-8}.
As shown in Table \ref{table:encoder_compare_pretrain}, we receive mix results for various pretraining schedules: no pretraining, pretraining on in-house data and pretraining on multilingual data.
It is mentioned by \cite{irie2019language} in language modeling that the \textit{Base} architecture works better than the \textit{Large}.
However, in acoustic modeling in \glsxtrshort{ASR}, our results prove against this statement.
Considering the amount of parameters between \textit{Base} and \textit{Large}\textsubscript{1-8}, 97M vs. 118M, we recommend the use of \textit{Base} in order to keep the performance competitive to \textit{Large}\textsubscript{1-8} while reducing the number of trainable parameters.

%\begin{table}[!h]
\centering
\begin{adjustbox}{width=\columnwidth,center}
\begin{tabular}{|c|c|c|c|c|c|c|} 
\hline
\multirow{3}{*}{Architecture} & \multirow{3}{*}{Features} & \multicolumn{2}{c|}{Pretraining}                                                                                               & Fine-tuning             & \multicolumn{2}{c|}{WER [\%]}  \\ 
\cline{3-7}
                              &                           & \multirow{2}{*}{Data (hours)}                                                                        & \multirow{2}{*}{Epochs} & \multirow{2}{*}{Epochs} & \multicolumn{2}{c|}{Viet.}     \\ 
\cline{6-7}
                              &                           &                                                                                                      &                         &                         & dev  & test                    \\ 
\hline
\textit{Base}                         & \multirow{2}{*}{Raw WF}   & \multirow{3}{*}{None}                                                                                & \multirow{3}{*}{None}   & \multirow{7}{*}{33}     & 35.8 & 39.9                    \\ 
\cline{1-1}\cline{6-7}
\textit{Large}\textsubscript{1-8}                        &                           &                                                                                                      &                         &                         & 35.0 & 40.7                    \\ 
\cline{1-2}\cline{6-7}
\textit{Large}\textsubscript{1-8}                        & GT                        &                                                                                                      &                         &                         & 36.8 & 38.4                    \\ 
\cline{1-4}\cline{6-7}
\textit{Base}                         & \multirow{16}{*}{Raw WF}  & \multirow{4}{*}{\begin{tabular}[c]{@{}c@{}}Respective~\\in-house \\dataViet. \\(0.01h)\end{tabular}} & \multirow{4}{*}{1}      &                         & 30.6 & 35.2                    \\
                              &                           &                                                                                                      &                         &                         &      &                         \\ 
\cline{1-1}\cline{6-7}
\textit{Large}\textsubscript{1-8}                        &                           &                                                                                                      &                         &                         & 31.8 & 35.7                    \\
                              &                           &                                                                                                      &                         &                         &      &                         \\ 
\cline{1-1}\cline{3-7}
\textit{Base}                         &                           & \multirow{4}{*}{\begin{tabular}[c]{@{}c@{}}Respective~\\in-house \\dataViet. \\(219h)\end{tabular}}  & \multirow{12}{*}{300}   & \multirow{12}{*}{26}    & 30.2 & 33.3                    \\
                              &                           &                                                                                                      &                         &                         &      &                         \\ 
\cline{1-1}\cline{6-7}
\textit{Large}\textsubscript{1-8}                        &                           &                                                                                                      &                         &                         & 31.5 & 33.4                    \\
                              &                           &                                                                                                      &                         &                         &      &                         \\ 
\cline{1-1}\cline{3-3}\cline{6-7}
\textit{Base}                         &                           & \multirow{4}{*}{\begin{tabular}[c]{@{}c@{}}Multilingual \\in-house~\\(1168h)\end{tabular}}           &                         &                         & 26.2 & 28.8                    \\
                              &                           &                                                                                                      &                         &                         &      &                         \\ 
\cline{1-1}\cline{6-7}
\textit{Large}\textsubscript{1-8}                        &                           &                                                                                                      &                         &                         & 26.8 & 28.7                    \\
                              &                           &                                                                                                      &                         &                         &      &                         \\ 
\cline{1-1}\cline{3-3}\cline{6-7}
\textit{Base}                         &                           & \multirow{4}{*}{\begin{tabular}[c]{@{}c@{}}Augmented~\\in-house \\dataViet. \\(1168h)\end{tabular}}  &                         &                         & 31.6 & 32.2                    \\
                              &                           &                                                                                                      &                         &                         &      &                         \\ 
\cline{1-1}\cline{6-7}
\textit{Large}\textsubscript{1-8}                        &                           &                                                                                                      &                         &                         & 31.0 & 32.3                    \\
                              &                           &                                                                                                      &                         &                         &      &                         \\
\hline
\end{tabular}
\end{adjustbox}
\caption{\glspl{WER} {[}\%{]} for models using unsupervised pre-training and fine-tuning from scratch. All fine-tunings use the \textit{Large}\textsubscript{1-8} or \textit{Base} architecture and are trained until full convergence on the in-house data of the respective language. Pre-trainings have been done with random initialization. Pre-training "None" means fine-tuning from scratch. Features used are Raw Waveform (Raw WF) or Gammatone (GT) \cite{schlueter:icassp07}.}
\label{table:encoder_feature_compare}
\end{table}
%\begin{table}[!h]
\centering
\begin{adjustbox}{width=\columnwidth,center}
\begin{tabular}{|c|c|c|c|c|c|c|} 
\hline
\multirow{2}{*}{Architecture} & \multirow{2}{*}{Features} & \multicolumn{2}{c|}{Pretraining}              & Fine-tuning         & \multicolumn{2}{c|}{WER [\%]}  \\ 
\cline{3-7}
                              &                           & Data (hours)          & Epochs                & Epochs              & dev  & test                    \\ 
\hline
\textit{Base}                 & \multirow{2}{*}{Raw WF}   & \multirow{2}{*}{None} & \multirow{2}{*}{None} & \multirow{2}{*}{33} & 35.8 & 39.9                    \\ 
\cline{1-1}\cline{6-7}
\textit{Large}\textsubscript{1-8}             &                           &                       &                       &                     & 35.0 & 40.7                    \\
\hline
\end{tabular}
\end{adjustbox}
\caption{\glspl{WER} {[}\%{]} for models fine-tuning from scratch. All fine-tunings use the \textit{Large}\textsubscript{1-8} or \textit{Base} architecture and are trained until full convergence on the Vietnamese in-house data. Pre-training "None" means fine-tuning from scratch. Features used are Raw Waveform (Raw WF).}
\label{table:encoder_compare_RawWF}
\end{table}

\subsection{Initialization comparison}

\begin{table}[!h]
\captionsetup{font=Large}
\centering
\begin{adjustbox}{width=\columnwidth,center}
\begin{tabular}{|c|c|c|c|c|c|c|} 
\hline
\multirow{2}{*}{Architecture} & \multirow{2}{*}{Init. scheme}  & \multicolumn{3}{c|}{Pretraining}                                                & \multicolumn{2}{c|}{WER [\%]}  \\ 
\cline{3-7}
                              &                                & Data                    & Hours                 & Epochs                & Hykist dev & Hykist test       \\ 
\hline
\textit{Base}                 & \multirow{2}{*}{Kaiming Init.} & \multirow{2}{*}{Viet. in-house} & \multirow{2}{*}{0.01} & \multirow{2}{*}{1}    & 30.6       & 35.2              \\ 
\cline{1-1}\cline{6-7}
\textit{Large}\textsubscript{1-8}             &                                &                                 &                       &                       & 31.8       & 35.7              \\ 
\hline
\textit{Base}                 & \multirow{2}{*}{Glorot Init.}  & \multirow{2}{*}{None}           & \multirow{2}{*}{-}    & \multirow{2}{*}{None} & 35.8       & 39.9              \\ 
\cline{1-1}\cline{6-7}
\textit{Large}\textsubscript{1-8}             &                                &                                 &                       &                       & 35.0       & 40.7              \\
\hline
\end{tabular}
\end{adjustbox}
\caption{\center{WERs for architecture \textit{Base} and \textit{Large}\textsubscript{1-8} using 2 different initialization schemes: 
\\Kaiming Initialization (in Fairseq framework \cite{facebook2019fairseq}) and Glorot Initialization (in RETURNN framework \cite{doetsch2016returnn}).}}
\label{table:encoder_compare_shortPretrain}
\end{table}

In the case of super short pretraining (1 epoch pretraining on only 0.01h of data), the results outperform those of raw waveform from scratch for both \textit{Base} and \textit{Large}\textsubscript{1-8} architecture as shown in Table \ref{table:encoder_compare_shortPretrain}.
The reason for the improvement comes from the difference of initialization schemes.
The parameters from the pretrained model are first initialized by Fairseq \cite{facebook2019fairseq} using Kaiming Initialization \cite{He_2015_ICCV}, and then fed into RETURNN \cite{doetsch2016returnn}, while the parameters for raw waveform training are initialized directly by RETURNN using Glorot (also known as Xavier) Initialization \cite{glorot2010understanding}.
We therefore recommend the use of Kaiming Initialization for \glsxtrshort{Wav2vec 2.0} architecture.

%\begin{table}[!h]
\centering
\begin{adjustbox}{width=\columnwidth,center}
\begin{tabular}{|c|c|c|c|c|c|c|} 
\hline
\multirow{2}{*}{Architecture} & \multirow{2}{*}{Features} & \multicolumn{2}{c|}{Pretraining}                                                                                        & Fine-tuning         & \multicolumn{2}{c|}{WER [\%]}  \\ 
\cline{3-7}
                              &                           & Data (hours)                                                                                     & Epochs               & Epochs              & dev  & test                    \\ 
\hline
\textit{Base}                 & \multirow{2}{*}{Raw WF}   & \multirow{2}{*}{\begin{tabular}[c]{@{}c@{}}Augmented~ in-house \\dataViet. (1168h)\end{tabular}} & \multirow{2}{*}{300} & \multirow{2}{*}{26} & 31.6 & 32.2                    \\ 
\cline{1-1}\cline{6-7}
\textit{Large}\textsubscript{1-8}             &                           &                                                                                                  &                      &                     & 31.0 & 32.3                    \\
\hline
\end{tabular}
\end{adjustbox}
\caption{\glspl{WER} {[}\%{]} for models fine-tuning using augmented data for pre-training.}
\label{table:encoder_compare_augment}
\end{table}
%\begin{table}[!h]
\centering
\begin{adjustbox}{width=\columnwidth,center}
\begin{tabular}{|c|c|c|c|c|c|c|} 
\hline
\multirow{2}{*}{Architecture} & \multirow{2}{*}{Features} & \multicolumn{2}{c|}{Pretraining} & Fine-tuning & \multicolumn{2}{c|}{WER [\%]}  \\ 
\cline{3-7}
                              &                           & Data (hours) & Epochs            & Epochs      & dev  & test                    \\ 
\hline
\textit{Large}\textsubscript{1-8}             & GT                        & None         & None              & 33          & 36.8 & 38.4                    \\
\hline
\end{tabular}
\end{adjustbox}
\caption{\glspl{WER} {[}\%{]} for models fine-tuning using \glsxtrfull{GT} feature. The setup is built the same to \glsxtrshort{CNN} feature.}
\label{table:feature_compare}
\end{table}