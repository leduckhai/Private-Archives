\chapter{Experiments}
\label{ch: Experiments}


\section{Data}

The first difficulty faced during the research in the HYKIST project is the lack of medical telephone speech dataset.
Having a small medical dataset - HYKIST, we therefore use HYKIST only for the recognition and use in-house non-medical telephone speech dataset for training. 
This poses a challenge to reach a high-performance ASR because of the mismatch in training and recognition datasets.
In addition, real-life dataset like HYKIST is difficult to be accurately transcribed by ASR models because of background noises, variation of speaking speed, unfamiliar pronunciation of medical terms...


\subsection{HYKIST data}

Our HYKIST project partner Triaphon recorded conversations between three people: a patient, a doctor, and an interpreter. 
The patient communicates in the non-German language - Arabic or Vietnamese - while the doctor communicates in German. 
The interpreter is fluent in both languages and assists the patient and doctor in communicating. 
In HYKIST, we have unique accents, foreign-born accents, from both interpreter and patient sides.
This directly makes HYKIST more difficult for machines and humans to transcribe, leading understandable bad recognition performance.
We received the audio recordings and had our transcribers perform speech transcription within the recordings. 
We divide the audio data into two sets: dev and test, with no speaker overlap between the two.

The data statistics for the dev and test sets for each individual language can be seen in Table \ref{table:data_stats}.
We only have a limited amount of data because we create it ourselves.
Furthermore, the number of speakers is limited, resulting in a low level of diversity in the testing data.
This may result in over-optimization of the evaluation data.
To address the impact of the data issues, we obtained additional training data from our industry partner Apptek and other sources.

\begin{table}[!h]
\captionsetup{font=Large}
\centering
\begin{adjustbox}{width=0.9\columnwidth,center}
\begin{tabular}{|l|c|c|c|c|c|c|c|} 
\hline
Language                    & Dataset                 & Usage                                                     & \# Spks & Hours  & Domain                                                                       & \begin{tabular}[c]{@{}c@{}}In-domain \\match\end{tabular} & \begin{tabular}[c]{@{}c@{}}Diversity \\level\end{tabular}  \\ 
\hline
Arabic                      & In-house                & pretr.                                                    & 3379    & 786    & Tel., Conv.                                                                  & Medium                                                    & Medium                                                     \\ 
\hline
German                      & In-house                & pretr.                                                    & 1723    & 177    & Tel., Conv.                                                                  & Medium                                                    & Medium                                                     \\ 
\hline
\multirow{5}{*}{Vietnamese} & In-house                & \begin{tabular}[c]{@{}c@{}}pretr., \\finetu.\end{tabular} & 2240    & 219    & Tel., Conv.                                                                  & Medium                                                    & Medium                                                     \\ 
\cline{2-8}
                            & \multirow{3}{*}{HYKIST} & adapt                                                     & 1       & 1      & \multirow{3}{*}{\begin{tabular}[c]{@{}c@{}}Tel., Conv.,\\Med.~\end{tabular}} & \multirow{3}{*}{High}                                     & \multirow{4}{*}{Low}                                       \\ 
\cline{3-5}
                            &                         & dev                                                       & 3       & 3      &                                                                              &                                                           &                                                            \\ 
\cline{3-5}
                            &                         & test                                                      & 2       & 2      &                                                                              &                                                           &                                                            \\ 
\cline{2-7}
                            & YouTube                 & pretr.                                                    & -       & 1.204  & Read books                                                                   & Low                                                       &                                                            \\ 
\hline
\multirow{2}{*}{Multi}      & In-house*               & pretr.                                                    & 7342    & 1.182  & Tel., Conv.                                                                  & Medium                                                    & \multirow{2}{*}{High}                                      \\ 
\cline{2-7}
                            & XLSR-53                 & pretr.                                                    & -       & 56.000 & Various                                                                      & Low                                                       &                                                            \\
\hline
\end{tabular}
\end{adjustbox}
\caption{\center{*The multilingual in-house training dataset is the combination of the Arabic, German and Vietnamese ones listed above. Domain: Telephone (Tel.), Conversational (Conv.), Medical (Med.). Sampling rate: 8kHz}}
\label{table:data_stats}
\end{table}

\subsection{In-house data}

AppTek, an industry partner, supplied us with annotated 8kHz conversational telephone speech data.
The audio data was collected during telephone conversations between customers and various call centers.
Table \ref{table:data_stats} displays the data statistics for the training sets for each of the three languages.
We can see that the amount of training data available varies between languages.

We also have speakers with accents and/or dialects for the Arabic and Vietnamese data.
For the Arabic data, we have four different datasets with distinct dialects: Syrian, Lebanese, Gulf, and Egyptian.
Besides, our Vietnamese dataset has dominantly 2 accents, Northern and Central Vietnamese, and a very small fraction of Southern Vietnamese accent.
The speakers with accents in the Vietnamese data are combined into a single dataset.

\subsection{YouTube}

We collected Vietnamese audio data from \glsxtrfull{YT} under Fair Use Policies\footnote{\href{https://support.google.com/youtube/answer/9783148}{https://support.google.com/youtube/answer/9783148}} in addition to our annotated datasets.
The domain in question is purely read speech, such as podcasts, audiobooks, radio stories, or something similar.
Pre-processing was done manually by removing non-speech parts such as music and noise, leaving only speech.
The audio files were then divided into 10-30 second segments.
Table \ref{table:data_stats} displays the data statistics for the web scraped data.
During data collection, we headed to the balance of accents and genders.
Therefore, the dataset is divided into Northern and Southern accents, yielding four subsets: Northern Female (518h), Northern Male (213h), Southern Female (290h) and Southern Male (183h).

%\subsection{VLSP}

%The VLSP dataset is a spontaneous-reading speech dataset provided by Association for Vietnamese Language and Speech Processing.
%The 2021 version\footnote{\href{https://vlsp.org.vn/vlsp2021/eval/asr}{https://vlsp.org.vn/vlsp2021/eval/asr}} includes 280 hours of transcribed data in the general domain and 400 hours of untranscribed data.
%The dataset is recorded in different real scenarios e.g., meeting conversation, lecture speech. 
%To evaluate our models, we use \cite{Duy_Khanh_Finetune_Wav2vec_2_0_2022}'s model which was trained on 100h of VLSP dataset, together with CommonVoice and VIVOS described below.

\subsection{CommonVoice Vietnamese}

We obtain the Vietnamese dataset from the massively-multilingual speech corpus  \cite{ardila2020commonvoice}. 
We use the data version 9.0\footnote{\href{https://commonvoice.mozilla.org/en/datasets}{https://commonvoice.mozilla.org/en/datasets}}, which includes 17 hours of noisy read speech data recorded by the large number of volunteer speakers.
The dataset is split into train/dev/test set. 
We evaluate our models by directly recognizing on dev and test sets.

\subsection{VIVOS}

VIVOS \cite{vivos_dataset} is a clean Vietnamese read speech corpus consisting of 15 hour recordings.
We obtain the dataset\footnote{\href{https://ailab.hcmus.edu.vn/vivos}{https://ailab.hcmus.edu.vn/vivos}} split into train/test sets.
We evaluate our models by directly recognizing on test set.
The test set includes 19 speakers and 48 minutes of duration in total.


\subsection{Monolingual text data}

Apptek, our project partner, provided monolingual text data for all three languages.
Text from various sources is included in the data.
The number of running words for each language is shown in Table \ref{table:LM_stats}.

\subsection{Domain}

As shown in Table \ref{table:data_stats} the data spans several domains.
The HYKIST project's target domain is medical conversational telephone speech.
The training data does not cover this specific domain.
This domain mismatch in our data is highlighted.
By listening to the audios and comparing them to our target domain, we can determine the in-domain match and diversity level.

