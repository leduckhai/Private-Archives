\begin{frame}{Effectiveness of Intermediate Focal Loss}
\begin{itemize}
    \item Improvement on HYKIST data:
\end{itemize}
% \usepackage[normalem]{ulem}
% \usepackage{multirow}


\begin{table}[!ht]
\captionsetup{font=Large}
\centering
\begin{adjustbox}{width=0.7\columnwidth,center}
\begin{tabular}{|c|c|c|c|c|c|c|} 
\hline
\multirow{2}{*}{Arch.}             & \multirow{2}{*}{Init.}             & \multicolumn{2}{c|}{Pre-training}                            & \multirow{2}{*}{With IF} & \multicolumn{2}{c|}{WER [\%]}  \\ 
\cline{3-4}\cline{6-7}
                                   &                                    & Data                                 & Hours                 &                          & Hykist dev & Hykist test       \\ 
\hline
\multirow{6}{*}{\textit{Large}\textsubscript{1-8}} & \multirow{4}{*}{None}              & \multirow{2}{*}{None}                & \multirow{2}{*}{-}    & No                       & 35.6       & 40.7              \\ 
\cline{5-7}
                                   &                                    &                                      &                       & Yes                      & 33.0       & 38.8              \\ 
\cline{3-7}
                                   &                                    & \multirow{6}{*}{Viet. in-house}      & \multirow{6}{*}{219}  & No                       & 30.4       & 33.4              \\ 
\cline{5-7}
                                   &                                    &                                      &                       & Yes                      & 28.6       & 33.0              \\ 
\cline{2-2}\cline{5-7}
                                   & \multirow{2}{*}{\textit{XLSR-53 }} &                                      &                       & No                       & 25.5       & 29.1              \\ 
\cline{5-7}
                                   &                                    &                                      &                       & Yes                      & 24.7       & 29.1              \\ 
\cline{1-2}\cline{5-7}
\multirow{2}{*}{\textit{Base}}     & \multirow{6}{*}{None}              &                                      &                       & No                       & 30.2       & 33.3              \\ 
\cline{5-7}
                                   &                                    &                                      &                       & Yes                      & 29.0       & 33.0              \\ 
\cline{1-1}\cline{3-7}
\multirow{6}{*}{\textit{Large}\textsubscript{1-8}} &                                    & \multirow{2}{*}{Viet. YT}            & \multirow{6}{*}{1168} & No                       & 29.8       & 35.2              \\ 
\cline{5-7}
                                   &                                    &                                      &                       & Yes                      & 26.1       & 30.8              \\ 
\cline{3-3}\cline{5-7}
                                   &                                    & \multirow{2}{*}{Viet. in-house + YT} &                       & No                       & 25.3       & 27.2              \\ 
\cline{5-7}
                                   &                                    &                                      &                       & Yes                      & 24.5       & 27.1              \\ 
\cline{2-3}\cline{5-7}
                                   & \multirow{2}{*}{\textit{XLSR-53}}  & \multirow{2}{*}{Viet. YT}            &                       & No                       & 24.3       & 28.1              \\ 
\cline{5-7}
                                   &                                    &                                      &                       & Yes                      & 23.4       & 28.1              \\
\hline
\end{tabular}
\end{adjustbox}
\caption{
    \center{Improvements of WERs on HYKIST data between pretraining schedules when applying IF Loss.
    All models are finetuned until full convergence on Vietnamese in-house data.}}
\label{table:if_loss_hykist_pos}
\end{table}
\end{frame}


\begin{frame}{Effectiveness of Intermediate Focal Loss}
\begin{itemize}
    \item Improvement on HYKIST data:
    \begin{itemize}
        \item 7/9 experiments of IF Loss experience total improvements, compared to only 3/9 experiments of ICE Loss.
        In addition, all WERs of IF Loss experiments are lower than those of ICE Loss experiments, except the one on HYKIST test set of from scratch training.
        \\ \textrightarrow \,
        When finetuning and recognizing on the same telephone domain, IF Loss works better than ICE Loss.
        
        \item WERs reduce from (29.8\% and 35.2\%) to (26.1\% and 30.8\%) for YT, from (30.4\% and 33.4\%) to (28.6\% and 33.0\%) for Vietnamese in-house, from (25.3\% and 27.2\%) to (24.5\% and 27.1\%) for in-house+YT.
        \\ \textrightarrow \,
        Effectiveness of IF Loss decreases when the pretrained data becomes more diverse.
        %\TODO{what could be the reason why IF helps in certain conditions?}
        
    \end{itemize}
\end{itemize}
\end{frame}


\begin{frame}{Effectiveness of Intermediate Focal Loss}
\begin{itemize}
    \item Degradation on HYKIST data:
\end{itemize}
% \usepackage[normalem]{ulem}
% \usepackage{multirow}


\begin{table}[!ht]
\captionsetup{font=Large}
\centering
\begin{adjustbox}{width=0.7\columnwidth,center}
\begin{tabular}{|c|c|c|c|c|c|} 
\hline
\multirow{2}{*}{Init.}            & \multicolumn{2}{c|}{Pre-training}                            & \multirow{2}{*}{With IF} & \multicolumn{2}{c|}{WER [\%]}  \\ 
\cline{2-3}\cline{5-6}
                                  & Data                                 & Hours                 &                          & Hykist dev & Hykist test       \\ 
\hline
\multirow{2}{*}{\textit{XLSR-53}} & \multirow{2}{*}{None}                & \multirow{2}{*}{-}    & No                       & 27.9       & 32.3              \\ 
\cline{4-6}
                                  &                                      &                       & Yes                      & 28.0       & 32.8              \\ 
\hline
\multirow{2}{*}{None}             & \multirow{2}{*}{Multiling. in-house} & \multirow{2}{*}{1168} & No                       & 26.8       & 28.7              \\ 
\cline{4-6}
                                  &                                      &                       & Yes                      & 25.2       & 29.3              \\
\hline
\end{tabular}
\end{adjustbox}
\caption{
    \center{Degradations of WERs on HYKIST data between pretraining schedules when applying IF Loss.
    All pretrainings use the \textit{Large}\textsubscript{1-8} architecture and are finetuned until full convergence on Vietnamese in-house data.}}
\label{table:if_loss_hykist_neg}
\end{table}
\begin{itemize}
    \item[] The degradation of \textit{XLSR-53} is rather small.
    Partial degradation in the multilingual in-house data but the average WER of dev and test set (25.2\% and 29.3\%) is even lower than the baseline (26.8\% and 28.7\%).
    \\ \textrightarrow \,
    In a rapid deployment of an ASR system, we recommend the direct use of IF Loss in training without the need of one more training without intermediate loss as a baseline for performance comparison.
\end{itemize}


\end{frame}


\begin{frame}{Effectiveness of Intermediate Focal Loss}
\begin{itemize}
    \item Improvement on CommonVoice and VIVOS data:
\end{itemize}
\begin{table}[!ht]
\centering
\begin{adjustbox}{width=0.9\columnwidth,center}
\begin{tabular}{|c|c|c|c|c|c|} 
\hline
\multirow{2}{*}{Arch.}             & \multirow{2}{*}{Pre-training data}                                              & \multirow{2}{*}{With IF} & \multicolumn{3}{c|}{WER [\%]}  \\ 
\cline{4-6}
                                   &                                                                                 &                          & CV dev & CV test & Vivos       \\ 
\hline
\multirow{4}{*}{\textit{Large}\textsubscript{1-8}} & \multirow{2}{*}{\begin{tabular}[c]{@{}c@{}}Viet. in-house\\(219h)\end{tabular}} & No                       & 16.4   & 35.6    & 31.3        \\ 
\cline{3-6}
                                   &                                                                                 & Yes                      & 15.8   & 34.5    & 29.6        \\ 
\cline{2-6}
                                   & \multirow{2}{*}{None}                                                           & No                       & 20.8   & 44.7    & 34.9        \\ 
\cline{3-6}
                                   &                                                                                 & Yes                      & 19.7   & 43.1    & 33.9        \\ 
\hline
\multirow{2}{*}{\textit{Base}}     & \multirow{2}{*}{\begin{tabular}[c]{@{}c@{}}Viet. in-house\\(219h)\end{tabular}} & No                       & 16.6   & 35.4    & 30.9        \\ 
\cline{3-6}
                                   &                                                                                 & Yes                      & 15.9   & 34.4    & 30.5        \\ 
\hline
\multirow{2}{*}{\textit{Large}\textsubscript{1-8}} & \multirow{2}{*}{\begin{tabular}[c]{@{}c@{}}Viet. YT\\(1168h)\end{tabular}}      & No                       & 16.4   & 34.4    & 28.7        \\ 
\cline{3-6}
                                   &                                                                                 & Yes                      & 14.5   & 30.9    & 26.9        \\
\hline
\end{tabular}
\end{adjustbox}
\caption{
    Improvements of \glspl{WER} {[}\%{]} on CommonVoice and VIVOS between pretraining schedules when applying \glsxtrshort{IF Loss}. All models are finetuned on Vietnamese in-house data. 
    Only 1 intermediate layer is applied in the middle \glsxtrshort{Transformer} block, e.g. position 4 for \textit{Large}\textsubscript{1-8} and 6 for \textit{Base} architecture.}
\label{table:if_loss_cvvivos_pos}
\end{table}
\end{frame}


\begin{frame}{Effectiveness of Intermediate Focal Loss}
\begin{itemize}
    \item Improvement on CommonVoice and VIVOS data:
    \begin{itemize}
        \item The ICE Loss makes 3/9 experiments totally improved (table Improvement on CommonVoice and VIVOS data), while the IF Loss makes 4/9.
        WERs for IF Loss on 3 read speech datasets are as competitive as ICE Loss.
        In addition, on the same telephone domain, IF Loss works better than ICE Loss.
        \\ \textrightarrow \,
        The IF Loss works better than the traditional ICE Loss in all domains.
    \end{itemize}
\end{itemize}
\end{frame}


\begin{frame}{Effectiveness of Intermediate Focal Loss}
\begin{itemize}
    \item Degradation on CommonVoice and VIVOS data:
    \begin{table}[!ht]
\centering
\begin{adjustbox}{width=0.9\columnwidth,center}
\begin{tabular}{|c|c|c|c|c|c|} 
\hline
\multirow{2}{*}{Init.}            & \multirow{2}{*}{Pre-training data}                                                    & \multirow{2}{*}{With IF} & \multicolumn{3}{c|}{WER [\%]}  \\ 
\cline{4-6}
                                  &                                                                                       &                          & CV dev & CV test & Vivos       \\ 
\hline
\multirow{4}{*}{\textit{XLSR-53}} & \multirow{2}{*}{None}                                                                 & No                       & 14.8   & 32.5    & 30.3        \\ 
\cline{3-6}
                                  &                                                                                       & Yes                      & 15.4   & 33.6    & 30.0        \\ 
\cline{2-6}
                                  & \multirow{2}{*}{\begin{tabular}[c]{@{}c@{}}Viet. in-house\\(219h)\end{tabular}}       & No                       & 11.5   & 29.4    & 27.2        \\ 
\cline{3-6}
                                  &                                                                                       & Yes                      & 13.0   & 29.8    & 27.6        \\ 
\hline
\multirow{4}{*}{None}             & \multirow{2}{*}{\begin{tabular}[c]{@{}c@{}}Multiling. in-house\\(1168h)\end{tabular}} & No                       & 15.2   & 29.7    & 29.5        \\ 
\cline{3-6}
                                  &                                                                                       & Yes                      & 14.5   & 30.6    & 28.1        \\ 
\cline{2-6}
                                  & \multirow{2}{*}{\begin{tabular}[c]{@{}c@{}}Viet. in-house + YT\\(1168h)\end{tabular}} & No                       & 12.9   & 26.5    & 21.0        \\ 
\cline{3-6}
                                  &                                                                                       & Yes                      & 12.7   & 28.6    & 22.1        \\ 
\hline
\multirow{2}{*}{\textit{XLSR-53}} & \multirow{2}{*}{\begin{tabular}[c]{@{}c@{}}Viet. YT\\(1168h)\end{tabular}}            & No                       & 11.8   & 28.4    & 25.6        \\ 
\cline{3-6}
                                  &                                                                                       & Yes                      & 13.2   & 29.1    & 24.5        \\
\hline
\end{tabular}
\end{adjustbox}
\caption{
    Degradations of \glspl{WER} {[}\%{]} on CommonVoice and VIVOS between pretraining schedules when applying \glsxtrshort{IF Loss}. All models are finetuned on Vietnamese in-house data. 
    Only 1 intermediate layer is applied in the middle \glsxtrshort{Transformer} block, e.g. position 4 for \textit{Large}\textsubscript{1-8} and 6 for \textit{Base} architecture.}
\label{table:if_loss_cvvivos_neg}
\end{table}
    \begin{itemize}
        \item Degradations in experiments pretrained on diverse data
        \\ \textrightarrow \,
        We recommend IF Loss only for less diverse pretrained data if the domain of finetuning and recognition data are too different.
    \end{itemize}
\end{itemize}
\end{frame}
