\begin{frame}{\textit{XLSR-53} as pretraining initialization}
    % \usepackage{multirow}

\begin{table}[!ht]
\captionsetup{font=Large}
\centering
\begin{tabular}{|c|c|c|c|c|c|} 
\hline
\multicolumn{4}{|c|}{Pre-training}                                                                                                                                & \multicolumn{2}{c|}{WER [\%]}  \\ 
\hline
Arch.                              & Data                                                                          & Hours                 & Epochs               & Hykist dev & Hykist test       \\ 
\hline
\multirow{2}{*}{\textit{Large}\textsubscript{1-8}} & None                                                                          & -                     & None                 & 27.6       & 31.9              \\ 
\cline{2-6}
                                   & \multirow{2}{*}{\begin{tabular}[c]{@{}c@{}}Viet. in-house\end{tabular}} & \multirow{2}{*}{219}  & 25                   & 27.6       & 29.5              \\ 
\cline{1-1}\cline{4-6}
\textit{Large}                     &                                                                               &                       & 100                  & 26.2       & 29.0              \\ 
\hline
\multirow{3}{*}{\textit{Large}\textsubscript{1-8}} & Viet. YT                                                                      & \multirow{3}{*}{1168} & \multirow{2}{*}{100} & 24.3       & 28.1              \\ 
\cline{2-2}\cline{5-6}
                                   & Viet. in-house + YT                                                           &                       &                      & 24.5       & 27.2              \\ 
\cline{2-2}\cline{4-6}
                                   & Multilingual in-house                                                         &                       & 50                   & 23.9       & 27.4              \\
\hline
\end{tabular}
\caption{\center{Table for unsupervised pre-training with the public \textit{XLSR-53} model as initialization. All finetunings use the \textit{Large}\textsubscript{1-8} architecture. 
The 1st model is the direct finetuning on Vietnamese in-house data, and the remaining models use \textit{XLSR-53} as initialization for pretrainings (full model \textit{Large} or cut-off model  \textit{Large}\textsubscript{1-8}). The chosen number of pretraining epochs is the best checkpoint.}}
\label{table: xlsr53_init_pretrain}
%\TODO{here multilingual is on par with monolingual, with less epochs.. I think the statement mono better than multi in this case cannot be made}
\end{table}
\end{frame}

\begin{frame}{\textit{XLSR-53} as pretraining initialization}
\begin{itemize}
    \item WERs of in-house Vietnamese data (31.4\% and 33.4\% in table Monolingual pretraining) vs. continued pretraining on in-house Vietnamese (27.6\% and 29.5\%) vs. direct finetuning with \textit{XLSR-53}\textsubscript{1-8} (27.6\% and 31.9\%).
    \\ \textrightarrow \, 
    Continued pretraining using \textit{XLSR-53} model outperforms the pretraining using random initialization and the direct finetuning using \textit{XLSR-53}.
    
    \item WERs of full \textit{XLSR-53} reduce from 27.6\% and 29.5\% to 26.2\% and 29.0\% 
    \\ \textrightarrow \,
    If the resource usage is neglected, the \textit{Large} model should be chosen for better accuracy instead of the cut-off \textit{Large}\textsubscript{1-8}. %\TODO{it is not clear if this is the larger model ot the higher number of epochs. what if we train the cut out model for 100 epochs??}
    
    \item WERs of multilingual pretraining reduce from 26.8\% and 28.7\% (table Multilingual pretraining) to 23.9\% and 27.4\%, those for YT reduce from 29.8\% and 35.2\% (table Monolingual pretraining) to 24.3\% and 28.1\%, those for in-house+YT reduce from 25.3\% and 27.2\% (table Monolingual pretraining) to 24.5\% and 27.2\% 
    \\ \textrightarrow \,
    Continued pretraining helps both the monolingual and the multilingual scenario. 
    However, the continued pretraining on less diverse data benefits more from the diverse and multilingual data.
\end{itemize}
\end{frame}