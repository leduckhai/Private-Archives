\begin{frame}{Encoder comparison}
    \begin{table}[!h]
\captionsetup{font=Large}
\centering
\begin{adjustbox}{width=0.8\columnwidth,center}
\begin{tabular}{|c|c|c|c|c|} 
\hline
\multirow{2}{*}{Architecture} & \multicolumn{2}{c|}{Pre-training}                              & \multicolumn{2}{c|}{WER [\%]}  \\ 
\cline{2-5}
                              & Data                                   & Hours                 & Hykist dev & Hykist test       \\ 
\hline
\textit{Base}                 & \multirow{2}{*}{None}                  & \multirow{2}{*}{-}    & 35.8       & 39.9              \\ 
\cline{1-1}\cline{4-5}
\textit{Large}\textsubscript{1-8}             &                                        &                       & 35.0       & 40.7              \\ 
\hline
\textit{Base}                 & \multirow{2}{*}{Viet. in-house}        & \multirow{2}{*}{219}  & 30.2       & 33.3              \\ 
\cline{1-1}\cline{4-5}
\textit{Large}\textsubscript{1-8}             &                                        &                       & 31.5       & 33.4              \\ 
\hline
\textit{Base}                 & \multirow{2}{*}{Multilingual in-house} & \multirow{2}{*}{1168} & 26.2       & 28.8              \\ 
\cline{1-1}\cline{4-5}
\textit{Large}\textsubscript{1-8}             &                                        &                       & 26.8       & 28.7              \\
\hline
\end{tabular}
\end{adjustbox}
\caption{\center{Comparison for \textit{Base} and \textit{Large}\textsubscript{1-8} using different pretraining schedules: no pretraining, pretraining on in-house data and on multilingual data.
All fine-tunings are done until full convergence on Vietnamese in-house data and the recognition is done on HYKIST.}}
\label{table:encoder_compare_pretrain}
\end{table}
\begin{itemize}
    \item WERs between \textit{Base} and \textit{Large}\textsubscript{1-8} fluctuate
    \\ \textrightarrow \,
    %\TODO{No! You cannot make this statement! This is not even the same task. Please show me where you saw this statement!}
    We recommend the use of \textit{Base} (97M parameters) in order to keep the performance competitive to \textit{Large}\textsubscript{1-8} (118M) while reducing the number of trainable parameters.
\end{itemize}
\end{frame}


\begin{frame}{Initialization comparison}
    \begin{table}[!h]
\captionsetup{font=Large}
\centering
\begin{adjustbox}{width=\columnwidth,center}
\begin{tabular}{|c|c|c|c|c|c|c|} 
\hline
\multirow{2}{*}{Architecture} & \multirow{2}{*}{Init. scheme}  & \multicolumn{3}{c|}{Pretraining}                                                & \multicolumn{2}{c|}{WER [\%]}  \\ 
\cline{3-7}
                              &                                & Data                    & Hours                 & Epochs                & Hykist dev & Hykist test       \\ 
\hline
\textit{Base}                 & \multirow{2}{*}{Kaiming Init.} & \multirow{2}{*}{Viet. in-house} & \multirow{2}{*}{0.01} & \multirow{2}{*}{1}    & 30.6       & 35.2              \\ 
\cline{1-1}\cline{6-7}
\textit{Large}\textsubscript{1-8}             &                                &                                 &                       &                       & 31.8       & 35.7              \\ 
\hline
\textit{Base}                 & \multirow{2}{*}{Glorot Init.}  & \multirow{2}{*}{None}           & \multirow{2}{*}{-}    & \multirow{2}{*}{None} & 35.8       & 39.9              \\ 
\cline{1-1}\cline{6-7}
\textit{Large}\textsubscript{1-8}             &                                &                                 &                       &                       & 35.0       & 40.7              \\
\hline
\end{tabular}
\end{adjustbox}
\caption{\center{WERs for architecture \textit{Base} and \textit{Large}\textsubscript{1-8} using 2 different initialization schemes: 
\\Kaiming Initialization (in Fairseq framework \cite{facebook2019fairseq}) and Glorot Initialization (in RETURNN framework \cite{doetsch2016returnn}).}}
\label{table:encoder_compare_shortPretrain}
\end{table}
    
\begin{itemize}
    \item For super short pretraining (1 epoch pretraining on only 0.01h of data), the results outperform those of raw waveform from scratch for both \textit{Base} and \textit{Large}\textsubscript{1-8} architecture
    \\ \textrightarrow \,
    In our experiments Kaiming Initialization shows better performance but the small differences in pretraining might lead to a less clear comparison.
    %\TODO{this is a very general statement. Which you want to prove with one experiment on one dataset which already has problems since the pre-training is performed differently $\rightarrow$ In our experiments Kaiming Initialization showed better performance but the differences in pre-training lead to a less clear comparison. }
\end{itemize}

%\TODO{please drop Fairseq and RETURNN specifier}
%\TODO{do you have Kaiming Init in RETURNN framework? so completely without pre-training?}
\end{frame}